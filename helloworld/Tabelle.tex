\documentclass[a4paper,10pt]{scrartcl}

\usepackage[utf8]{inputenc}
\usepackage[ngerman]{babel}
\usepackage[T1]{fontenc}
\usepackage{lmodern}

%Für Grafiken
\usepackage{graphicx}

%Für besondere Aufzählungen
\usepackage{paralist}

\title{Tabellen und Aufzählungen}
\author{Latex Tutorial}
\date{\today}

\begin{document}

\maketitle
\tableofcontents
\newpage

\section{Aufzählungen}
%Für Aufzählungen stehen unter LaTeX die Umgebungen itemize, description und enumerate zur Verfügung. Innerhalb der Aufzählung wird jedes neue Element mit dem Befehl \item eingeleitet. Selbstverständlich können Aufzählungen auch ineinandergeschachtelt werden.
		
\subsection{Typen}

\begin{description}
\item{Erster Punkt}
\item{Zweiter Punkt}
\item{Dritter Punkt}
\end{description}

\begin{itemize}
\item{Erster Punkt}
\item{Zweiter Punkt}
\item{Dritter Punkt}
\end{itemize}
		
\begin{enumerate}
\item{Erster Punkt}
\item{Zweiter Punkt}
\item{Dritter Punkt}
\end{enumerate}

\subsection{Besondere Aufzählungen}

%Mit \usepackage{paralist} lässt sich die Art der Aufzählung ändern:
%
    %\begin{compactenum}[(i)] %führt zu (i), (ii), (iii), (iv), ...
    %\begin{compactenum}[(I)] %führt zu (I), (II), (III), (IV), ...
    %\begin{compactenum}[a)] %führt zu a), b), c), d), ...

\begin{compactenum}[(i)]
		\item{Erster Punkt}
		\item{Zweiter Punkt}
		\item{Dritter Punkt}
\end{compactenum}

\begin{compactenum}[(I)]
		\item{Erster Punkt}
		\item{Zweiter Punkt}
		\item{Dritter Punkt}
\end{compactenum}

\begin{compactenum}[a)]
		\item{Erster Punkt}
		\item{Zweiter Punkt}
		\item{Dritter Punkt}
\end{compactenum}

\subsection{Verschachtelte Aufzählungen}

%~\par sorgt dafür das, dass vorangestellte Item wie so eine kleine Überschrift verwendet wird. Löscht es einfach mal, dann seht ihr was passiert!
\begin{description}
   \item[Nummerierte Aufzählung]~\par
   \begin{enumerate}
      \item Weitere Aufzählung
      \begin{enumerate}
         \item erstens.
         \item zweitens.
      \end{enumerate}
      \item Toter Punkt
   \end{enumerate}
	
   \item[Nichtnummerierte Aufzählung]~\par
   \begin{itemize}
      \item Unteraufzählung
      \begin{itemize}
         \item erstens.
         \item zweitens.
      \end{itemize}
      \item der andere Punkt.
   \end{itemize}
\end{description}

\newpage
\section{Tabellen}
%Alles über Tabellen könnt Ihr hier nochmal nachlesen: http://en.wikibooks.org/wiki/LaTeX/Tables (Auf Englisch)


%c 	Zentrierter Text
%l 	linksbündiger Text
%r 	rechtsbündiger Text

\begin{tabular}{ l c r }
  1 & 2 & 3 \\
  4 & 5 & 6 \\
  7 & 8 & 9 \\
\end{tabular}



\begin{tabular}{ l | c || r }
  1 & 2 & 3 \\
  4 & 5 & 6 \\
  7 & 8 & 9 \\
\end{tabular}



\begin{tabular}{ l | c || r }
  \hline                        
  1 & 2 & 3 \\
  4 & 5 & 6 \\
  7 & 8 & 9 \\
  \hline  
\end{tabular}



\begin{center}
  \begin{tabular}{ l | c || r }
    \hline
    1 & 2 & 3 \\ \hline
    4 & 5 & 6 \\ \hline
    7 & 8 & 9 \\
    \hline
  \end{tabular}
\end{center}



\begin{tabular}{|r|l|}
  \hline
  7C0 & hexadecimal \\
  3700 & octal \\ \cline{2-2}
  11111000000 & binary \\
  \hline \hline
  1984 & decimal \\
  \hline
\end{tabular}

\newpage
\subsection{Breite für Spalten festlegen}

Ohne festgelegte Breite:
\begin{center}
    \begin{tabular}{| l | l | l | l |}
    \hline
    Day & Min Temp & Max Temp & Summary \\ \hline
    Monday & 11C & 22C & A clear day with lots of sunshine.
    However, the strong breeze will bring down the temperatures. \\ \hline
    Tuesday & 9C & 19C & Cloudy with rain, across many northern regions. Clear spells
    across most of Scotland and Northern Ireland,
    but rain reaching the far northwest. \\ \hline
    Wednesday & 10C & 21C & Rain will still linger for the morning.
    Conditions will improve by early afternoon and continue
    throughout the evening. \\
    \hline
    \end{tabular}
\end{center}

Mit festgelegter Breite:
\begin{center}
    \begin{tabular}{ | l | l | l | p{5cm} |}
    \hline
    Day & Min Temp & Max Temp & Summary \\ \hline
    Monday & 11C & 22C & A clear day with lots of sunshine.  
    However, the strong breeze will bring down the temperatures. \\ \hline
    Tuesday & 9C & 19C & Cloudy with rain, across many northern regions. Clear spells
    across most of Scotland and Northern Ireland,
    but rain reaching the far northwest. \\ \hline
    Wednesday & 10C & 21C & Rain will still linger for the morning.
    Conditions will improve by early afternoon and continue
    throughout the evening. \\
    \hline
    \end{tabular}
\end{center}

\newpage
\subsection{Legende in der Tabelle}

\begin{table}[htb]
\begin{tabular}{|r|l|}
  \hline
  7C0 & hexadecimal \\
  3700 & octal \\ \cline{2-2}
  11111000000 & binary \\
  \hline \hline
  1984 & decimal \\
  \hline
\end{tabular}
\caption{Eine super interessante Tabelle}
\label{tab:supertabelle}
\end{table}

\end{document}